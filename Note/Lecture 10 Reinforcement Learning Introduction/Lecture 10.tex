\documentclass{ctexart}
\usepackage{Note}
\begin{document}
\section{强化学习与多臂老虎机}
\subsection{强化学习引言}
人类是通过与环境互动来学习的,而\tbf{强化学习(Reinforcement Learning, RL)},就是通过计算来实现从互动中学习.它是一种目标导向的互动学习,\textit{互动性}是它与监督学习的重要区别,而有无明确的目标则是它与无监督学习的主要区别.强化学习需要学习的主要内容,就是怎样做才能使回报最大化.它有两个显著特征:\textit{试错}与\textit{延迟回报}.\\
\indent 强化学习的主体通常称作\tbf{智能体(Agent)},也通常称作代理.我们将在下一节介绍智能体的数学框架,即感知(Sensation, $S$),行动(Action, $A$)和目标(Goal ,$G$).
\subsection{多臂老虎机:探索与利用的平衡}
\subsubsection{多臂老虎机问题的定义}
有关多臂老虎机的现实背景不再赘述.这里用数学语言描述多臂老虎机问题:\\
\indent 智能体在每一步都需要从$K$个可能的行动$\{a_k\}_{k=1}^{K}$中选择一个行动,当选择行动$a_k$后将得到回报$R_k$, $R_k$的静态分布$p(R_k)$仅取决于行动$a_k$.智能体应当怎样逐步选择$N$个行动$A_t(t=1,\cdots,N)$使得获得的总回报$\displaystyle\sum_{t=1}^{N}R_t$最大?\\
\indent 定义行动$a$的价值函数为
\[q_\ast(a):=\mathbb{E}[R_t|A_t=a]\]
如果$q_\ast(a)$已知,那么上述问题就很简单,只需在每步选择使得$q_\ast(a)$最大的$a$即可.然而,智能体实际上不能获知$q_\ast(a)$,只能在第$t$步做选择时根据之前的结果计算$q_\ast(a)$的估计值$Q_t(a)$,并据此选择合适的行动.\\
\indent 如果选择$Q_t(a)$更大的行动,那么主要是利用当前对行动价值函数的了解进行回报最大化,但是容易陷入局部最优中;如果选择其它行动,那么更有利于精确地估计行动价值函数$Q_t(a)$,但是容易损失回报.这两者经常是矛盾的,倾向一者就会远离另一者,这就是多臂老虎机中探索-利用权衡的难题.\\
\indent 针对上述问题,人们提出了许多算法.
\subsubsection{多臂老虎机的若干算法}
多臂老虎机的算法大都基于对行动价值函数的估计,即估计行动$a$的价值$Q_t(a)$并在此基础上采取行动.行动价值函数的一种简单的估计方法是统计此前所有采取$a$行动后得到回报的均值,即
\[Q_t(a)=\dfrac{\displaystyle\sum_{i=1}^{t-1}R_iI(A_i=a)}{\displaystyle\sum_{i=1}^{t-1}I(A_i=a)}\]
其中$I$是示性函数.当然,采用贝叶斯方法通过后验概率估计也是可行的.
\paragraph{贪心算法}
在智能体采取行动时可以采用\tbf{贪心算法(Greedy Algorithm)},即选取使得$Q_t(a)$取最大值的行动,即
\[A_t=\arg\max_{a}Q_t(a)\]
为了防止贪心算法陷入局部最优,可以强制加入探索的成分,典型的是\tbf{$\bs\ep$-贪心算法},即有小概率$\ep$随机选择所有可能的行动,大概率$1-\ep$遵循贪心算法选择回报期望最高的行动.\\
\indent 然而,在$K$个选择的$\ep$-贪心算法中,在足够长的时间后,尽管$Q_t(a)$已经足够接近$q_\ast(a)$,算法仍然有$\dfrac{K-1}{K}\ep$的概率选择非最优的步骤.这启示我们也许可以在不同的阶段采取不同的策略.
\paragraph{UCB算法}
多臂老虎机问题的著名算法,\tbf{置信区间上界算法(Upper Confidence Bound Algorithm, UCB)}在一定程度上解决了上述问题,即随着行动数目$t$的增大而逐步减小探索的概率.\\
\indent UCB算法分析了$Q_t(a)$估计$q_\ast(a)$的误差.通常,对随机变量$x$进行$n$次测量,结果为$\li X,n$,那么通常把结果写成
\[x=\bar{X}\pm\dfrac{\sigma}{\sqrt{n}},\quad \bar{X}=\dfrac{1}{n}\sum_{i=1}^{n}X_i,\sigma=\sqrt{\dfrac{1}{n-1}\sum_{i=1}^{n}(X_i-\bar{X})^2}\]
根据这一分析结果,我们认为$q_\ast(a)$有很大概率落在以下区间(即置信区间)内:
\[\left[Q_t(a)-\dfrac{\sigma_t(a)}{\sqrt{N_t(a)}},Q_t(a)+\dfrac{\sigma_t(a)}{\sqrt{N_t(a)}}\right]\]
其中$N_t(a)=\sum_{i=1}^{t-1}I(A_i=a)$是行动$a$被采纳的次数. UCB算法采用了一种面对不确定性时的乐观想法,将上述置信区间的上界作为$q_\ast(a)$的估计,给出如下的行动结果:
\[A_t=\arg\max_{a}\left[Q_t(a)+c\sigma_t(a)\sqrt{\dfrac{\ln t}{N_t(a)}}\right]\]
引入因子$\ln t$和常数$c$可以更好地控制误差.通过选择具有最高置信上界的行动$a$,就会倾向于选择那些既有较高期望奖励又较少被探索的行动,从而获得更好的探索-利用平衡效果.
\end{document}