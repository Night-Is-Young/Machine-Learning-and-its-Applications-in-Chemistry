\documentclass{ctexart}
\usepackage{Note}
\begin{document}
\setcounter{FormalCounter}{0}
\section{核方法:近邻法与支持向量机}
\subsection{密度估计的非参数法}
对于随机变量$\vec{x}$,如果知道其概率密度函数$p(\vec{x})$,就能够计算出其在某一区域$R$内取值的概率.现在,我们需要根据$\vec{x}$的一组采样点$\mathcal{D}:\left\{\vec{x}_n\right\}_{n=1}^{N}$估计概率密度函数$p(\vec{x})$.这就是\tbf{密度估计(density estimation)}问题.
\begin{definition}[密度估计]
    给定随机变量$\vec{x}$的一组采样点$\mathcal{D}:\left\{\vec{x}_n\right\}_{n=1}^{N}$,密度估计是根据$\mathcal{D}$估计随机变量$\vec{x}$的概率密度函数$p(\vec{x})$的过程.
\end{definition}
密度估计有两类办法:
\begin{enumerate}
    \item \tbf{参数法}:假定$p(\vec{x})$具有已知的带参数形式,那么问题转化为应用最大似然法确定参数的值.
    \item \tbf{非参数法}:不对$p(\vec{x})$作任何假设,而是直接根据样本$\mathcal{D}$估计$p(\vec{x})$.
\end{enumerate}
\indent 本节介绍的直方图方法,核密度估计法和近邻法都属于非参数法.
\subsubsection{直方图方法}
直方图是我们熟知的表示随机变量分布的方法.将数据空间划分为若干个小区域,称为\tbf{区间(bins)}或\tbf{箱(buckets)},然后统计每个区间内的样本点数目,最后通过归一化得到概率密度函数的估计.具体地,设数据空间被划分为$M$个区间$\left\{\mathcal{R}_j\right\}_{j=1}^{M}$,每个区间的体积为$V$,则在区间$\mathcal{R}_j$内的概率密度函数估计为
\[p_{\mathcal{R}_j}(\vec{x})=\frac{1}{NV}\sum_{n=1}^{N}\mathbb{I}(x_n\in \mathcal{R}_j)=\dfrac{n_j}{NV}\]
其中$n_j$即$\mathcal{R}_j$中的样本点的数目.\\
\indent 直方图的数学原理可以推导如下.
\begin{proof}
    根据概率密度函数的定义,随机变量$\vec{x}$落在某一区域$\mathcal{R}$内的概率为
    \[P=\int_{\mathcal{R}}p(\vec{x})\di\vec{x}\]
    对$\vec{x}$随机采样$N$次,有$K$个点落入$\mathcal{R}$的概率服从二项分布:
    \[p\left(K|N,P\right)=C_{N}^{K}P^K(1-P)^{N-K}\]
    如果$N$和$K$都很大,那么上述二项分布是一个窄峰,我们就可以近似地认为$P\approx \dfrac{K}{N}$.另一方面,如果区域$\mathcal{R}$足够小,那么可以认为在该区域内$p(\vec{x})$近似为常数,即$P=p(\vec{x})V$.结合上述两式即可得
    \[p(\vec{x})=\dfrac{K}{NV}\]
    这就说明了直方图方法的合理性.
\end{proof}
直方图方法的优点是简单直观,但缺点也很明显:结果曲线不光滑,并且高维空间下将因维度灾难而效果变差.
\subsubsection{核密度估计法}
\subsubsection{近邻法}
\subsection{核方法的主要思想}
\subsection{支持向量机}
支持向量机通过核方法进行非线性分类.它的主要想法是:在所有可能的分类超平面中,选择一个使得分类间隔最大的超平面作为最终的分类超平面.
\subsubsection{支持向量机的数学原理}
我们用更严谨的语言描述上述问题.考虑线性分类模型
\[y(\vec{x})=\vec{w}^{}\]
\end{document}