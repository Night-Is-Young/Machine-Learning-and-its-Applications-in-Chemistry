\documentclass{ctexart}
\usepackage{Note}
\begin{document}
\setcounter{FormalCounter}{0}
\section{分类的线性方法}
\subsection{线性拟合与感知器}
既然分类的输出是离散值,它当然也属于连续值.于是,我们可以简单地沿用线性回归方法,即
\[y=g(\vec{x};\vec{w})=w_0x_0+\cdots+w_Mx_M=\vec{w}^{\text{t}}\vec{x}\]
回归模型输出的是连续值,因此需要离散化.例如,当输出值为$\{0,1\}$时,可以使用
\[g(\vec{x})=\begin{cases}
    1,&f(\vec{x};\vec{w})\geq \dfrac12\\
    0,&f(\vec{x};\vec{w})<\dfrac12
\end{cases}\]
将结果离散化.然而,如果出现远离数据聚集处的点,那么线性回归可能会因为迎合这些点而产生较大偏差.因此,我们可以直接用阶跃函数进行拟合,即
\[y=g(\vec{x};\vec{w})=f\left(\vec{w}^{\text{t}}\vec{x}\right)=H\left(\vec{w}^{\text{t}}\vec{x}\right),\ \ \text{where}\ H(x)=\begin{cases}
    1,&x\geq 0\\
    0,&x<0
\end{cases}\]
其中$f(z)$为激活函数,对输入的$z$进行非线性变换得到结果.这里采用的激活函数为阶跃函数$H(z)$.直接用上述定义的$g(\vec{x};\vec{w})$进行拟合(例如进行最大似然法估计等),这就是\tbf{感知器(perceptron)}模型.
\begin{definition}[感知器]
    使用阶跃函数$H(z)$作为激活函数的线性分类模型,即
    \[y=g(\vec{x};\vec{w})=H\left(\vec{w}^{\text{t}}\vec{x}\right)\]
    称为\textbf{感知器(perceptron)}模型.
\end{definition}
\subsection{逻辑回归}
\subsubsection{逻辑回归的模型}
感知器模型的激活函数是阶跃函数,它在$x=0$处不可导,因此无法使用梯度下降法等涉及导数的方法进行优化.为了替换成光滑的函数以便优化,我们可以用sigmoid函数$\sigma(z)$替换前面的阶跃函数:
\[y=g\left(\vec{x};\vec{w}\right)=\sigma\left(\vec{w}^{\text{t}}\vec{x}\right),\ \ \text{where}\ \sigma(z)=\dfrac{1}{1+e^{-z}}\]
这就是\tbf{逻辑回归(logistic regression)}模型.
\begin{definition}[逻辑回归]
    使用sigmoid函数$\sigma(z)$作为激活函数的线性分类模型,即
    \[y=g\left(\vec{x};\vec{w}\right)=\sigma\left(\vec{w}^{\text{t}}\vec{x}\right),\ \ \text{where}\ \sigma(z)=\dfrac{1}{1+e^{-z}}\]
    称为\textbf{逻辑回归(logistic regression)}模型.
\end{definition}
上述函数返回一个$(0,1)$间的连续值.因此,我们需要将结果解读为样本属于某一类别的概率.\\
\indent 当$\sigma\left(\vec{w}^{\text{t}}\vec{x}\right)>0.5$时,我们预测样本属于类别$1$的概率更大;否则预测样本属于类别$0$的概率更大.所预测的两种类别的边界称作\tbf{决策面(Decision Boundary)},由下式描述:
\[\vec{w}^{\text{t}}\vec{x}=0\]
这是$\vec{x}$的空间中的一个线性平面.
\subsubsection{逻辑回归的求解}
与前面一样,逻辑回归也可以归结为一个概率优化问题,使用极大似然法即可.仍然假定数据集为$\mathcal{D}=\left\{\vec{x}_n,t_n\right\}_{n=1}^{N}$,我们的目标是最大化概率
\[\max_{\vec{w}}P\left(\right)\]
\end{document}