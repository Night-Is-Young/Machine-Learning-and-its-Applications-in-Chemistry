\documentclass{ctexart}
\usepackage{Note}
\begin{document}
\setcounter{FormalCounter}{0}
\section{分类的线性方法}
\subsection{线性拟合与感知器}
既然分类的输出是离散值,它当然也属于连续值.于是,我们可以简单地沿用线性回归方法,即
\[y=g(\vec{x};\vec{w})=w_0x_0+\cdots+w_Mx_M=\vec{w}^{\text{t}}\vec{x}\]
回归模型输出的是连续值,因此需要离散化.例如,当输出值为$\{0,1\}$时,可以使用
\[g(\vec{x})=\begin{cases}
    1,&f(\vec{x};\vec{w})\geq \dfrac12\\
    0,&f(\vec{x};\vec{w})<\dfrac12
\end{cases}\]
将结果离散化.然而,如果出现远离数据聚集处的点,那么线性回归可能会因为迎合这些点而产生较大偏差.因此,我们可以直接用阶跃函数进行拟合,即
\[y=g(\vec{x};\vec{w})=f\left(\vec{w}^{\text{t}}\vec{x}\right)=H\left(\vec{w}^{\text{t}}\vec{x}\right),\ \ \text{where}\ H(x)=\begin{cases}
    1,&x\geq 0\\
    0,&x<0
\end{cases}\]
其中$f(z)$为激活函数,对输入的$z$进行非线性变换得到结果.这里采用的激活函数为阶跃函数$H(z)$.直接用上述定义的$g(\vec{x};\vec{w})$进行拟合(例如进行最大似然法估计等),这就是\tbf{感知器(perceptron)}模型.
\begin{definition}[感知器]
    使用阶跃函数$H(z)$作为激活函数的线性分类模型,即
    \[y=g(\vec{x};\vec{w})=H\left(\vec{w}^{\text{t}}\vec{x}\right)\]
    称为\textbf{感知器(perceptron)}模型.
\end{definition}
\subsection{逻辑回归}
\subsubsection{逻辑回归的模型}
感知器模型的激活函数是阶跃函数,它在$x=0$处不可导,因此无法使用梯度下降法等涉及导数的方法进行优化.为了替换成光滑的函数以便优化,我们可以用sigmoid函数$\sigma(z)$替换前面的阶跃函数:
\[y=g\left(\vec{x};\vec{w}\right)=\sigma\left(\vec{w}^{\text{t}}\vec{x}\right),\ \ \text{where}\ \sigma(z)=\dfrac{1}{1+e^{-z}}\]
这就是\tbf{逻辑回归(logistic regression)}模型.
\begin{definition}[逻辑回归]
    使用sigmoid函数$\sigma(z)$作为激活函数的线性分类模型,即
    \[y=g\left(\vec{x};\vec{w}\right)=\sigma\left(\vec{w}^{\text{t}}\vec{x}\right),\ \ \text{where}\ \sigma(z)=\dfrac{1}{1+e^{-z}}\]
    称为\textbf{逻辑回归(logistic regression)}模型.
\end{definition}
上述函数返回一个$(0,1)$间的连续值.因此,我们需要将结果解读为样本属于某一类别的概率.\\
\indent 当$\sigma\left(\vec{w}^{\text{t}}\vec{x}\right)>0.5$时,我们预测样本属于类别$1$的概率更大;否则预测样本属于类别$0$的概率更大.所预测的两种类别的边界称作\tbf{决策面(Decision Boundary)},由下式描述:
\[\vec{w}^{\text{t}}\vec{x}=0\]
这是$\vec{x}$的空间中的一个线性平面.
\subsubsection{逻辑回归的求解}
与前面一样,逻辑回归也可以归结为一个概率优化问题,使用极大似然法即可.仍然假定数据集为$\mathcal{D}=\left\{\vec{x}_n,t_n\right\}_{n=1}^{N}$.当$t_n=1$时,模型预测结果与$t_n$一致的概率时$y_n=\sigma(\vec{w}^\t\vec{x}_n)$,反之则为$1-y_n$.综合而言,模型预测正确的概率为
\[p(t_n|\vec{w})=y_n^{t_n}(1-y_n)^{1-t_n}\]
假定数据点之间是独立的,似然函数可以写成
\[p(\vec{t}|\vec{w})=\prod_{n=1}^{N}y_n^{t_n}(1-y_n)^{1-t_n}\]
定义交叉熵误差函数
\[E(\vec{w})=-\ln p(\vec{t}|\vec{w})=-\sum_{n=1}^{N}\left[t_n\ln y_n+(1-t_n)\ln(1-y_n)\right]\]
这样做是因为取对数后的函数是凸函数,可以更好地进行求解.为了利用梯度下降法求解$\vec{w}$的最佳值,我们先来推导误差函数的梯度$\nabla_\vec{w}E(\vec{w})$.首先有
\[\dfrac{\di\sigma(z)}{\di z}=\dfrac{-\e^{-z}}{-(1+\e^{-z})^2}=\sigma(z)[1-\sigma(z)]\]
于是
\[\begin{aligned}
    \nabla_{\vec{w}}E(\vec{w})
    &=-\sum_{n=1}^{N}\left(\dfrac{t_n}{y_n}-\dfrac{1-t_n}{1-y_n}\right)\dfrac{\di y_n}{\di\vec{w}}\\
    &=-\sum_{n=1}^{N}\left(\dfrac{t_n}{y_n}-\dfrac{1-t_n}{1-y_n}\right)\dfrac{\di \sigma(\vec{w}^\t\vec{x}_n)}{\di\vec{w}}\\
    &=-\sum_{n=1}^{N}\left(\dfrac{t_n}{y_n}-\dfrac{1-t_n}{1-y_n}\right)y_n(1-y_n)\vec{x}_n\\
    &=\sum_{n=1}^{N}(y_n-t_n)\vec{x}_n
\end{aligned}\]
这一矢量给出了$E(\vec{w})$随$\vec{w}$增加最快的方向和速度.因此,梯度下降的迭代求解可以写为
\[\vec{w}^{(\tau+1)}=\vec{w}^{(\tau)}-\eta\nabla_\vec{w}E(\vec{w})\]
当数据个数太多时,梯度计算公式中的求和将耗费大量时间.考虑到迭代过程本身就是一个逐步逼近最优解的过程,每次的移动方向不一定要很精确,因此可以利用部分数据而非全部数据计算梯度.在极端条件下,每次只利用一个训练数据点,即
\[\vec{w}^{(\tau+1)}=\vec{w}^{(\tau)}-\eta\nabla_\vec{w}E_n(\vec{w}),\quad E_n(\vec{w})=(y_n-t_n)\vec{x}_n\]
\subsubsection{基函数和正则化的使用}
基函数和正则化的使用与在线性回归中是类似的,这里不多赘述.引入基函数可以更好地描述决策面的形状,它在基函数的线性空间中是一个平面,但在样本数据空间中则可以表现地更为复杂.
\subsubsection{多分类问题}
处理多分类问题常有以下两种方法.
\begin{definition}[一对余方法]
    \tbf{一对余方法(one-versus-the-rest)}的基本想法是每次考虑将样本分为一个类别和其余类别,从而将多分类问题转化为多个二分类问题.
\end{definition}
一对余方法适用于分类不互斥的方法.
\begin{definition}[softmax方法]
    \tbf{softmax}方法对每个类别$i$定义
    \[z_i(\vec{x})=\vec{w}^\t_{(i)}\vec{x}\]
    然后利用如下softmax函数计算$\vec{x}$属于类别$i$的概率
    \[y^{(i)}(\vec{x})=\dfrac{\e^{z_i}}{\displaystyle\sum_{i}\e^{z_i}}=\dfrac{\e^{\vec{w}_{(i)}^\t\vec{x}}}{\displaystyle\sum_i\e^{\vec{w}_{(i)}^\t\vec{x}}}\]
\end{definition}
softmax方法满足各类别概率之和等于$1$的要求,因此特别适用于类别互斥的情形.
\end{document}