\documentclass{ctexart}
\usepackage{Note}
\begin{document}
\section{概率图模型}
\subsection{有向图:贝叶斯网络}
\begin{definition}[贝叶斯网络]
    \tbf{贝叶斯网络(Bayesian network)},又称\tbf{信念网络(Belief Network)},是一种概率图模型,其网络拓朴结构是一个有向无环图,边的终点所指的事件依赖于边的起点所指的事件.
\end{definition}
\subsubsection{贝叶斯网络的三种情形}
讨论贝叶斯网络中事件的条件独立性时,总会遇到以下三种情形:
\begin{enumerate}[label=\tbf{\arabic*.}]
    \item \tbf{尾对尾(tail-to-tail)}:两根箭头的尾部相连:
    \[a\leftarrow c\rightarrow b\]
    可以简单地理解为$a,b$有共同的起因$c$(尽管并不一定是因果关系).根据贝叶斯网络的性质,此时的联合概率为
    \[p(a,b,c)=p(a|c)p(b|c)p(c)\]
    于是
    \[p(a,b)=\sum_{c}p(a|c)p(b|c)p(c)\]
    一般而言,上式并不等于
    \[p(a)p(b)=\left(\sum_{c}p(a|c)p(c)\right)\left(p(b|c)p(c)\right)\]
    因此$a$与$b$并不独立,记作$a\not\!\perp\!\!\!\perp b\mid \varnothing$.这里$\varnothing$表示空集,即没有前提条件.\\
    当固定(已知)$c$时则有
    \[p(a,b|c)=\dfrac{p(a,b,c)}{p(c)}=p(a|c)p(b|c)\]
    即$a$与$b$在$c$固定的条件下是独立的:
    \[a\perp\!\!\!\perp b\mid c\]
    可以举一个简单的例子:假定$a,b,c$分别是一所小学中学生的鞋子尺码,阅读能力和年龄.当$c$不固定时,将会得出阅读能力和鞋子尺码正相关的结论,但是当固定年龄$c$时就会发现这两者是独立的.
    \item \tbf{头对尾(head-to-tail)}:两根箭头头尾相连:
    \[a\rightarrow c\rightarrow b\]
    此时$a,b,c$的联合概率为
    \[p(a,b,c)=p(a)p(c|a)p(b|c)\]
    如果$c$不固定,则
    \[p(a,b)=p(a)\sum_{c}p(c|a)p(b|c)\]
    一般而言,上式并不等于
    \[p(a)p(b)=p(a)\sum_{a}\left(p(a)\sum_{c}p(c|a)p(b|c)\right)\]
    因此$a\not\!\perp\!\!\!\perp b\mid \varnothing$.当固定$c$时
    \[p(a,b|c)=\dfrac{p(a,b,c)}{p(c)}=\dfrac{p(a)p(c|a)p(b|c)}{p(c)}=\dfrac{p(a,c)p(b|c)}{p(c)}=p(a|c)p(b|c)\]
    即$a\perp\!\!\!\perp b\mid c$.也即,固定$c$之后切断了$a$对$b$的影响.
    \item \tbf{头对头(head-to-head)}:两根箭头的头部相连:
    \[a\rightarrow c\leftarrow b\]
    此时$a,b,c$的联合概率为
    \[p(a,b,c)=p(a)p(b)p(c|a,b)\]
    如果$c$不固定,则有
    \[p(a,b)=p(a)p(b)\sum_{c}p(c|a,b)=p(a)p(b)\]
    即$a\perp\!\!\!\perp b\mid\varnothing$.然而,如果$c$固定,则有
    \[p(a,b|c)=\dfrac{p(a,b,c)}{p(c)}=\dfrac{p(a)p(b)p(c|a,b)}{\displaystyle\sum_{a,b}p(a)p(b)p(c|a,b)}\]
    而
    \[p(a|c)p(b|c)=\dfrac{\displaystyle p(a)p(b)\left(\sum_{a}p(a)p(c|a,b)\right)\left(\sum_{b}p(b)p(c|a,b)\right)}{\displaystyle\left(\sum_{a,b}p(a)p(b)p(c|a,b)\right)^2}\]
    一般情况下两者并不相等,即$a\not\!\perp\!\!\!\perp b\mid c$.这表明在两个事件共同的结果确定时,这两者具有一定的相关关系.\\
    可以举一个简单的例子:假定$a,b,c$分别是文化课成绩好,体育成绩好和进入某大学.如果文化课成绩好和体育成绩好都能进入大学,那么观察大学里的学生就很可能发现文化课成绩好的体育成绩差,反之亦然.实际上两者并没有负相关关系,但是固定$c$之后就导致了表面上的负相关.
\end{enumerate}
\subsection{朴素贝叶斯}
\subsubsection{朴素贝叶斯分类器}
贝叶斯分类器是一种用于分类任务的特殊的贝叶斯网络。它假设在类别$\mathcal{C}$固定时,输入$\vec{x}$的各个分量(特征)的分布之间是独立的.朴素指的就是这种算法假定各个分量的分布独立.尽管真实数据可能不满足这个简化假设,但这一方法对很多复杂的问题还是很有效果的.\\
\indent 在朴素贝叶斯模型中,类别$\mathcal{C}$是父节点,输入$\vec{x}$的各个分量$\li x,D$是其子节点,子节点之间没有连线.于是联合概率分布为
\[p(\vec{x}|\mathcal{C})=p(\mathcal{C})\prod_{i=1}^{D}p_i(x_i|\mathcal{C})\]
于是就把分布处理成多个一元分布,然后根据数据对每个维度进行训练.这很好地避免了维度灾难的问题.\\
\indent 朴素贝叶斯的推断可以利用贝叶斯公式,即
\[p(\mathcal{C}|\vec{x})=\dfrac{p(\vec{x}|\mathcal{C})p(\mathcal{C})}{p(\vec{x})}=\dfrac{p(\mathcal{C})}{p(\vec{x})}\prod_{i=1}^{D}p_i(x_i|\mathcal{C})\]
\subsection{马尔科夫随机场与玻尔兹曼机}
\subsubsection{马尔科夫随机场}
\begin{definition}[马尔科夫随机场]
    \tbf{马尔科夫随机场(Markov random field)},又称\tbf{马尔科夫网络(Markov network)},也是一类概率图模型,属于无向图模型(undirected graphical model),连接结点的边是没有方向的.
\end{definition}
\subsubsection{玻尔兹曼机与受限玻尔兹曼机}
\end{document}