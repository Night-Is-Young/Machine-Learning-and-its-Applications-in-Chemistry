\documentclass{ctexart}
\usepackage{Note}
\begin{document}
\section{混合模型}
\subsection{$K$均值法}
\subsubsection{$K$均值法的原理}
假设我们需要把数据集$\mathcal{D}:\{\vec{x}_n\}$中的数据点分到$K$个组.直观地讲,我们希望决定一个点分到某个组中时,该点与该组中其他点的距离尽可能近,而与其他组中点的距离尽可能远.于是我们需要找到能衡量这一距离的指标.\\
\indent 聚类分析的一种思路是寻找一些\tbf{原型点(Prototype points)}$\{\bs\mu_k\}$代表每个组.计算任一数据点$\vec{x}_n$与第$k$个组的距离时,只需计算$||\vec{x}_n-\bs\mu_k||$即可. $K$均值法就采用这样的思路,用二乘误差衡量模型的误差,就得到下面的\tbf{畸变函数(Distorition function)}:
\[J(\mat{R},\{\bs\mu_k\})=\sum_{n=1}^{N}\sum_{k=1}^{K}\mat{R}_{nk}||\vec{x}_n-\bs\mu_k||^2\]
其中$\mat{R}_{nk}$是第$n$个数据点分组结果的\tbf{独热编码},如果$\vec{x}_n$被分到第$k$组,则$\mat{R}_{nk}=1$,否则$\mat{R}_{nk}=0$.于是,聚类问题就转化为如下最小化问题:
\[\arg\min_{\mat{R},{\bs\mu_k}}J(\mat{R},\{\bs\mu_k\})\]
畸变函数$J$类似于监督学习中的误差函数.\\
\indent 求解上述最小化问题时,我们可以采用\tbf{交替优化(Alternating optimization)}的方法,具体步骤为:
\begin{enumerate}[label=\tbf{\arabic*.}]
    \item 固定$\{\bs\mu_k\}$,求$\mat{R}$使得$J(\mat{R},\{\bs\mu_k\})$最小化.不难看出,只需对于每个数据点$\vec{x}_n$找到与其距离最近的原型点$\bs\mu_k$,并将$\vec{x}_n$分到对应的组中(即令$\mat{R}_{nk}=1$)即可.
    \item 固定$\mat{R}$,求$\{\bs\mu_k\}$使得$J(\mat{R},\{\bs\mu_k\})$最小化.对每个原型点$\bs\mu_k$,不难看出当其取聚类中所有点的均值时,畸变函数$J$取得最小值.即:
    \[\bs\mu_k=\dfrac{\displaystyle\sum_{n=1}^{N}\mat{R}_{nk}\vec{x}_n}{\displaystyle\sum_{n=1}^{N}\mat{R}_{nk}}\]
    这也是$K$均值法名称的由来.
    \item 不断重复步骤1和2,直到畸变函数$J$收敛(即不再发生变化).可以证明$J$在此过程中单调下降,但不一定收敛到全局最小值.因此,实际应用中通常需要多次运行$K$均值算法,每次重新随机初始化,并选择畸变函数$J$最小的结果作为最终结果.
\end{enumerate}
\indent 对于在线学习的情形,我们可以采用\tbf{在线$K$均值算法(Online K-means algorithm)}.每次只处理一个数据点$\vec{x}_n$,并根据该点更新对应的原型点$\bs\mu_k$.具体地,对于每个数据点$\vec{x}_n$,我们首先找到与其距离最近的原型点$\bs\mu_k$,然后根据如下公式更新该原型点:
\[\bs\mu_k\leftarrow\bs\mu_k+\eta_n(\vec{x}_n-\bs\mu_k)\]
其中$\eta_n$是学习率,通常取$\eta_n=\dfrac{1}{s_k}$,其中$s_k$是到目前为止被分到第$k$组的数据点数量.这种更新方式等价于将$\bs\mu_k$更新为当前所有被分到第$k$组的数据点的均值.
\subsubsection{超参数$K$的选取}
不难看出$K$均值法只有一个超参数$K$.聚类作为非监督学习不能通过已知的标签辅助选择$K$.除去通过实际问题的情况选择$K$外,可以使用一些辅助指标选择$K$.一种常用的方法是\tbf{肘部法则(Elbow method)}.具体地,我们可以计算不同$K$值下畸变函数$J$的值,并绘制$J$随$K$变化的曲线.通常情况下,随着$K$的增加,畸变函数$J$会减小,但减小的幅度会逐渐变小.当曲线出现明显的“肘部”时,对应的$K$值就是一个较好的选择.
\subsection{高斯混合模型}
\subsubsection{高斯混合模型的原理}
$K$均值法的优势在于算法简单易行,执行高效,但它的主要缺点是它对于聚类中心平均值的使用太单一,倾向于认为组内数据的分布呈球形.这可能在某些数据分布复杂的情况下表现不佳.为了解决这个问题,我们可以使用更复杂的模型来描述数据的分布,例如\tbf{高斯混合模型(Gaussian Mixture Model,GMM)}.\\
\indent 高斯混合模型假设$\vec{x}$的分布是多个高斯分布的加权和.具体而言假设有$K$个高斯分布,每个高斯分布有其均值$\bs\mu_k$和协方差矩阵$\bs\Sigma_k$,以及对应的混合系数$\pi_k$,满足$\displaystyle\sum_{k=1}^{K}\pi_k=1$,则高斯混合模型的概率密度函数为:
\[p(\vec{x})=\sum_{k=1}^{K}\pi_k\mathcal{N}(\vec{x}|\bs\mu_k,\bs\Sigma_k)\]
每个高斯分布代表聚类的一个组别.引入隐藏变量$\vec{z}$,其第$k$个分量$\vec{z}_k$取$1$时表示$\vec{x}$属于第$k$个组别(对于给定的数据点而言就是独热编码).于是可知
\[p(\vec{z}_k=1)=\pi_k,\quad p(\vec{x}|\vec{z}_k=1)=\mathcal{N}(\vec{x}|\bs\mu_k,\bs\Sigma_k)\]
于是根据Bayes公式可知:
\[p(\vec{x})=p(\vec{x}|\vec{z})p(\vec{z})\]
\indent 现在我们考虑如何求解高斯混合模型的参数$\{\pi_k,\bs\mu_k,\bs\Sigma_k\}$.如果$\vec{z}$是已知的,那么可以很容易地根据多元高斯分布的性质求解$\pi_k,\bs\mu_k,\bs\Sigma_k$的值.然而实际上$\vec{z}$是未知的(这正是聚类的结果),因此我们需要使用\tbf{期望最大化(Expectation-Maximization,EM)}算法来估计参数.\\
\indent 简单而言,首先对于每个$\vec{x}_n$估计$\vec{z}_n$的值,然后基于${\vec{x}_n,\vec{z}_n}$求解模型参数;然后用求出的模型参数重新推断$\vec{z}_n$,如此迭代直到收敛为止.
\begin{enumerate}[label=\tbf{\arabic*.}]
    \item 初始化参数$\pi_k,\bs\mu_k,\bs\Sigma_k$.
    \item \tbf{E步骤}:根据猜测的$\vec{z}_n$,记$\gamma_k(\vec{z}_n)$为数据点$\vec{x}_n$属于第$k$个聚类的概率,则属于第$k$个聚类的数据数目为
    \[N_k=\sum_{n=1}^{N}\gamma_k(\vec{z}_n)\]
    基于多元高斯分布的性质可得
    \[\bs\mu_k=\dfrac{\displaystyle\sum_{n=1}^{N}\gamma_k(\vec{z}_n)\vec{x}_k}{\displaystyle\sum_{n=1}^{N}\gamma_k(\vec{z}_n)}=\dfrac{1}{N_k}\sum_{n=1}^{N}\gamma_k(\vec{z}_n)\vec{x}_n\]
    \[\bs\Sigma_k=\dfrac{\displaystyle\sum_{n=1}^{N}\gamma_k(\vec{z}_n)(\vec{x}_n-\bs\mu_k)(\vec{x}_n-\bs\mu_k)^\t}{\displaystyle\sum_{n=1}^{N}\gamma_k(\vec{z}_n)}=\dfrac{1}{N_k}\sum_{n=1}^{N}\gamma_k(\vec{z}_n)(\vec{x}_n-\bs\mu_k)(\vec{x}_n-\bs\mu_k)^\t\]
    \[\pi_k=\dfrac{N_k}{N}\]
    \item \tbf{M步骤}:基于更新后的参数,重新计算$\gamma_k(\vec{z}_n)$.根据Bayes公式可得
    \[\gamma_k(\vec{z}_n)=p(\vec{z}_k=1|\vec{x}_n)=\dfrac{p(\vec{x}_n|\vec{z}_{k}=1)p(\vec{z}_k=1)}{\displaystyle\sum_{j=1}^{K}\pi_j\mathcal{N}(\vec{x}_n|\bs\mu_j,\bs\Sigma_j)}=\dfrac{\pi_k\mathcal{N}(\vec{x}_n|\bs\mu_k,\bs\Sigma_k)}{\displaystyle\sum_{j=1}^{K}\pi_j\mathcal{N}(\vec{x}_n|\bs\mu_j,\bs\Sigma_j)}\]
    \item 重复E步骤和M步骤,直到参数收敛.
\end{enumerate}
\end{document}